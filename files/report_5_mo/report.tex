\documentclass{amsart}

\title{5 Month Interim Report}
\author{Jake Saunders}

\newcommand{\Z}{\mathbb{Z}}
\newcommand{\F}{\mathbb{F}}
\newcommand{\RP}{\mathbb{RP}}

\begin{document}
\maketitle

I am investigating whether the Euler genus can be realised as a map of highly structured ring spectra.

\section{Euler homology and spectra}

Let $\chi(M)$ denote the Euler characteristic of a manifold.
Unoriented bordism is a (co)homology theory represented by a spectrum denoted $MO$.
My current aim is to find a spectrum $EO$ and a map $e : MO \to EO$ of highly-structured ring spectra which descends to the (graded) Euler characteristic map when taking homotopy groups: $\pi_*(e) : \pi_*(MO) \to \pi_*(EO)$.
That is, the map
\[ [M] \mapsto \chi(M) \cdot t^{\dim M/2}. \]
A ring homomorphism with domain a bordism ring is known as a \textit{genus}, hence the name Euler genus.
A map of homology theories realising this was constructed in \cite{weber}.

The unoriented bordism ring is well-known to be the graded polynomial ring over $\F_2$ with a generator in each degree $n$ such that $n$ is not one less than a power of 2, i.e.\
\[ \pi_*(MO) = \F_2[x_2, x_4, x_5, x_6, x_8, \dots]. \]

I showed that the kernel of this Euler characteristic map is the ideal generated by elements
\begin{itemize}
    \item $x_{2k + 1}$ for $2k + 1 \neq 2^i - 1$,
    \item $[\RP^{2k}] - [\RP^2]^k$ for $k > 1$,
\end{itemize}
and furthermore that this is generated by a regular sequence.

If one has a commutative $S$-algebra such as $MO$ or $MU$ (the representing spectrum of complex cobordism), results of \cite{EKMM} allow us to define quotients by ideals generated by regular sequences, and in some cases give us highly structued ring spectra.
The highly-structued results, however, require that the $S$-algebra is \textit{even}, that is, its homotopy groups of odd degree vanish.
$MU$, however, has homotopy groups $\Z[a_2, a_4, a_6, \dots]$ (one polynomial generator in each positive even degree), and since complex manifolds are orientable and by \cite{orientable-mod-2}, orientable manifolds of dimension $4k + 2$ have vanishing mod 2 Euler characteristic, we can define a spectrum $EU$ which admits an associative, commutative $MU$-algebra structure, and such that the natural map $MU \to EU$ descends to the mod-2 Euler characteristic.
It appears that different results are required for $MO$.

\section{Stratifold homology and ad theories}
In between the first and second semesters, I re-read \cite{weber} and became interested in the construction of the Euler homology theory, which uses the theory of stratifolds introduced by Kreck \cite{kreck-diff-alg-top}.
I read \cite{kreck-zoo} which gives a summary of similar such theories.
Another paper \cite{minatta} uses stratifolds to realise the signature of a manifold as a homology theory, which is another bordism invariant.
This led me to read about Quinn spectra induced by \textit{bordism-type theories} introduced in \cite{quinn} and refined to the notion of an \textit{ad theory} in \cite{laures-mcclure-mult, laures-mcclure-comm}.
After discussing this with my supervisor, we established that this may be a route to solving the problem, albeit quite an overpowered and un-enlightening one, and so will focus on using different methods.

\section{Obstruction theory for $E_\infty$ structures}
In the last few weeks I have been reading about obstruction theories for highly structured ring spectra \cite{structured-ring-spectra}, called Hochschild homology, gamma homology, and topological Andre-Quillen homology.
In particular, the case I am working on is similar to the previously solved case of complex $K$-theory, and so I am studying the paper \cite{baker-richter}, where the gamma homology of $KU$ is shown to vanish, and therefore $KU$ has a unique $E_\infty$ structure.
I expect to be able to apply similar ideas to $EO$, and show that it also has a unique $E_\infty$ structure.

\section{Misc}
In the past 5 months, I have given three talks:
\begin{itemize}
    \item Sheaf Seminar: Matrix Algebras, Matrix Varieties and the Utility of Weyr Matrices (talk on my masters project).
    \item Galois Cohomology Reading Group: Groups with cohomological dimension bounded above by 1, and dualizing modules.
    \item Infinity Categories Reading Group: Defining Infinity Categories.
\end{itemize}
I am an assistant demonstrator for MAS005. I am also an assistant demonstrator for the AMSP Y12 Enrichment Sessions.

\bibliography{refs}
\bibliographystyle{siam}

\end{document}