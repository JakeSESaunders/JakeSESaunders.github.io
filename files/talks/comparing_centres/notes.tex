\documentclass{article}

\usepackage{amsmath, amsfonts, amsthm, amssymb}
\usepackage{quiver}
\usepackage{url}

\newcommand{\C}{\mathcal{C}}
\newcommand{\D}{\mathcal{D}}
\newcommand{\op}{{\tempop}}
\newcommand{\set}[1]{\left\lbrace #1 \right\rbrace}

\DeclareMathOperator{\Cat}{Cat}
\DeclareMathOperator{\Hom}{Hom}
\DeclareMathOperator{\Id}{Id}
\DeclareMathOperator{\Mod}{Mod}
\DeclareMathOperator{\Mon}{Mon}
\DeclareMathOperator{\Set}{Set}
\DeclareMathOperator{\tempop}{op}

\theoremstyle{definition}
\newtheorem{definition}{Definition}

\title{Comparing centres}
\author{Jacob Saunders}

\begin{document}

\maketitle

The notion of a \textit{centre}, e.g.\ the centre of a group, ring or monoid, is fundamental to algebra.
For concrete algebraic structures, the centre is defined as
\[
  Z(A) := \set{z \in A : az = za}.
\]
Applications include
\begin{itemize}
  \item checking Morita equivalence, since the centre of a ring is a Morita invariant property;
  \item defining $R$-algebras: if $A$ is a ring then an $R$-algebra structure on $A$ is equivalently a map $R \to Z(A)$;
  \item and much, much more!
\end{itemize}

This definition doesn't work if your objects aren't `sets with structure' (e.g.\ the homotopy category of topological spaces) or maybe your category is concretizable, but the structure doesn't obviously descend to subobjects in the way required for this definition.
There are a number of alternative definitions for the categorically minded.
Definitions and comparisons will be carried out using tools which are available in a variety of contexts (Yoneda lemma, tensor-hom adjunction, monadicity, ends).

\section{Categorical definitions of centre}
To avoid discussing concepts of enriched category theory, we will focus on the case of ordinary categories and the category of sets.
As such, in this section, $A$ is a monoid in $\Set$.

There are 3 handy criteria for any definition of centre (bearing in mind that the aim is to be able to replace sets with something else):
\begin{enumerate}
  \item a commutative multiplication;
  \item a universal property (biggest subobject such that...);
  \item simple criteria for its existence.
\end{enumerate}
Here are some candidate definitions.

\begin{definition}
  The \textit{limit centre} of $A$ is the equalizer
  \[
    Z(A) \dashrightarrow A \rightrightarrows [A, A],
  \]
  where the arrows $A \to [A, A]$ are adjunct to $\mu_A, \mu_A \circ \tau : A^2 \to A$.
\end{definition}

This satisfies 2 and 3 easily (it has the universal property of the limit, and it exists if your category has equalizers), but 1 is not so clear.

% TODO check this. I don't understand how A is the monoidal unit but Hom_{A|A}(A, A) isn't A.
For background on the following definition, see \cite{vitale}.
\begin{definition}
  The \textit{bimodule centre} of $A$ is the hom set $Z(A) := \Hom_{A|A}(A, A)$ (the homomorphisms from $A$ to itself when considered as an $A|A$-bimodule).
\end{definition}

This one satisfies 3, and 1 (by the Eckmann--Hilton argument, since $A$ is the monoidal unit).
Criterion 2 is less clear here.

Crystalised in the work of Kontsevich \cite[Claim 1]{kontsevich} and realised by Lurie \cite[5.3]{lurie}, the following definition works in a more general context than the others, so we fix a monoidal category $\C$ and an object $M$ of $\C$.
The definition requires a category of \textit{module structures}.

\begin{definition}
  Let $\Mod \C$ denote the category consisting of
  \begin{itemize}
    \item objects triples $(R, N, \rho : R \otimes N \to N)$, where $R$ is a monoid in $\C$ and $\rho$ is an $R$-module structure on $N$, and
    \item morphisms $(R, N, \rho) \to (R', N', \rho')$ given by pairs $(\phi : R \to R', f : N \to \phi^\ast N')$, where $\phi$ is a map of monoids and $f$ is a map of $R$-modules.
  \end{itemize}
\end{definition}

\begin{definition}
  Let $\C$ be a monoidal category and let $M \in \C$.
  The \textit{universal centre} of $M$, denoted $Z_\C(M)$, is the terminal object of the pullback below.
  % https://q.uiver.app/#q=WzAsNCxbMCwwLCJcXE1vZCAoTSkiXSxbMSwwLCJcXE1vZCBcXEMiXSxbMSwxLCJcXEMiXSxbMCwxLCJcXHNldHtNfSJdLFsxLDJdLFszLDJdLFswLDNdLFswLDFdLFswLDIsIiIsMSx7InN0eWxlIjp7Im5hbWUiOiJjb3JuZXIifX1dXQ==
  \[\begin{tikzcd}
    {\Mod (M)} & {\Mod \C} \\
    {\set{M}} & \C
    \arrow[from=1-1, to=1-2]
    \arrow[from=1-1, to=2-1]
    \arrow["\lrcorner"{anchor=center, pos=0.125}, draw=none, from=1-1, to=2-2]
    \arrow[from=1-2, to=2-2]
    \arrow[from=2-1, to=2-2]
  \end{tikzcd}\]
  Here, the vertical map sends a triple to its module, and the rightmost arrow includes the one object category consisting of $M$ and its identity map.
\end{definition}
The objects of $\Mod(M)$ are monoids in $\C$ together with a module structure on $M$.
Note that, if $\C$ has internal hom objects, then $Z_\C(M) = [M, M]$.
If we want to recover the centre of a monoid of sets, we let $\C$ be the \textbf{category of monoids} in $\Set$.

Again, the Eckmann--Hilton argument covers 1, and it has a very clear universal property.
It's not immediately clear how to check existence.

One other object traditionally referred to as the centre is the centre of a category.
This was originally introduced in \cite{butler-horrocks} for abelian categories, and extended to enriched categories in \cite{lindner}.

\begin{definition}
  Let $\C$ be a category.
  The \textit{centre} of $\C$ is the set of natural transformations $[\Id_\C, \Id_\C]$.
\end{definition}

This can be used to get a definition of centre of a monoid via the \textit{delooping} construction.

\begin{definition}
  Let $A$ be a monoid in $\Set$.
  The \textit{delooping} of $A$, denoted $BA$, is the category consisting of
  \begin{itemize}
    \item a single object $\bullet$,
    \item morphisms $\Hom_{BA}(\bullet, \bullet) = A$, with identity and composition law inherited from the multiplication of $A$.
  \end{itemize}
\end{definition}

It looks a bit random at first, but here is the justification: such a natural transformation $\eta$ has one component $\eta_\bullet : \bullet \to \bullet \in A$ which must satisfy the condition $\eta_\bullet \circ a = a \circ \eta_\bullet$, which inside $A$ translates to $\eta_\bullet$ commutes with $a$.

\section{Comparison of definitions}
Our strategy for comparing these definitions is to compare the centre of a monoid $M$ to the centre of the delooping category $BM$.

\subsection{The universal centre}
Our aim is to prove that $Z_{\Mon \Set}(A) \simeq [\Id_{BA}, \Id_{BA}]$.

Delooping extends to an adjunction:
\[
  B : \Mon \Set \rightleftarrows \Cat_* : \Omega,
\]
where $\Cat_*$ is the category of pointed categories.
Here, the left adjoint $B$ is fully faithful, and the right adjoint $\Omega$ sends a category to the monoid of endomorphisms of the distinguished object.
Using this adjunction, we can rephrase the desired statement as 
\[
  Z_{\Mon \Set}(M) \simeq \Omega[BM, BM] \simeq \Omega Z_{\Cat}(BM).
\]
That is, if we can understand how the centre construction interacts with adjunctions, then the statement is proved.
The diagram below commutes and the horizontal arrows are all equivalences.
% https://q.uiver.app/#q=WzAsNixbMCwwLCJbU3xBLCBSWihMQSl8QV0iXSxbMCwxLCJbQSwgQV0iXSxbMSwxLCJbQSwgUkxBXSJdLFsyLDEsIltMQSwgTEFdIl0sWzEsMCwiW1N8QSwgUlooTEEpfFJMQV0iXSxbMiwwLCJbTFMgfCBMQSwgWihMQSkgfCBMQV0iXSxbMSwyLCJcXGV0YV97QVxcYXN0fSJdLFsyLDMsIigtKV57XFxmbGF0fSJdLFswLDFdLFswLDQsIlxcZXRhX3tBXFxhc3R9Il0sWzQsMl0sWzUsM10sWzQsNSwiKC0pXntcXGZsYXR9Il1d
\[\begin{tikzcd}
	{[S|A, RZ(LA)|A]} & {[S|A, RZ(LA)|RLA]} & {[LS | LA, Z(LA) | LA]} \\
	{[A, A]} & {[A, RLA]} & {[LA, LA]}
	\arrow["{\eta_{A\ast}}", from=1-1, to=1-2]
	\arrow[from=1-1, to=2-1]
	\arrow["{(-)^{\flat}}", from=1-2, to=1-3]
	\arrow[from=1-2, to=2-2]
	\arrow[from=1-3, to=2-3]
	\arrow["{\eta_{A\ast}}", from=2-1, to=2-2]
	\arrow["{(-)^{\flat}}", from=2-2, to=2-3]
\end{tikzcd}\]
The monoid $RZ(LA)$ is the centre of $A$ if the fibre over $\Id_A$ is contractible.
The equivalences mean that this fibre is isomorphic to the fibre over $\Id_{LA}$, which is contractible because $Z(LA)$ is the centre of $LA$.

\subsection{The limit centre}
It's not immediately clear whether the set $[\Id_{\C}, \Id_{\C}]$ of all natural transformations posesses a universal property in such a way that we can exhibit it as a limit.
The key is in the word \textbf{all}.
In terms of components, such a natural transformation is a collection of functions $\eta_C : C \to C$ subject to the naturality condition $\eta_C \circ f = f \circ \eta_\C$.
Therefore the collection of \textbf{all} natural transformations is the collection of such functions
\[ \prod_{C \in \C} \Hom_\C(C, C) \]
subject to the condition that the precomposition
\begin{gather*}
  \Hom_\C(C, C) \xrightarrow{(-)^*} [\Hom_\C(C, D), \Hom_\C(C, D)],\\
  f \mapsto (g \mapsto g \circ f)
\end{gather*}
and postcomposition
\begin{gather*}
  \Hom_\C(D, D) \xrightarrow{(-)_*} [\Hom_\C(C, D), \Hom_\C(C, D)],\\
  f \mapsto (h \mapsto f \circ h),
\end{gather*}
agree.
In other words, this is an equalizer
\[ [\Id_\C, \Id_\C] \dasharrow \prod_{C \in \C} \Hom_\C(C, C) \rightrightarrows \prod_{C, D \in \C} [\Hom_\C(C, D), \Hom_\C(C, D)]. \]
If we specialize to the case where $\C = BA$, this reduces to
\[
  [\Id_{BA}, \Id_{BA}] \dasharrow A \rightrightarrows [A, A].
\]

\subsection{An aside on ends}
The previous explanation is a bit wordy, and relies on the natural transformation object being computed as an equalizer.
This is not true in a higher categorical context, as the natural transformations have to keep track of higher homotopies as well.

For this, it's useful to consider the intermediate notion of an \textit{end}.
Ends are traditionally introduced as representing objects for wedges, but this definition is inconvenient for other contexts.
Nevertheless, I'll use it as a segue for introducing my preferred representability definition.

If you have a functor $H : \C^\op \times \C \to \Set$, then a wedge from $X \in \Set$ to $H$ is a collection of maps $X \xrightarrow{\eta_C} H(C, C)$ such that  the diagram below commutes for every $f : C \to C'$.
% https://q.uiver.app/#q=WzAsNCxbMCwwLCJYIl0sWzEsMCwiSChDLCBDKSJdLFsxLDEsIkgoQywgQycpIl0sWzAsMSwiSChDJywgQycpIl0sWzMsMiwiZl4qIiwyXSxbMSwyLCJmXyoiXSxbMCwzLCJcXGV0YV97Qyd9IiwyXSxbMCwxLCJcXGV0YV9DIl1d
\[\begin{tikzcd}
	X & {H(C, C)} \\
	{H(C', C')} & {H(C, C')}
	\arrow["{\eta_C}", from=1-1, to=1-2]
	\arrow["{\eta_{C'}}"', from=1-1, to=2-1]
	\arrow["{f_*}", from=1-2, to=2-2]
	\arrow["{f^*}"', from=2-1, to=2-2]
\end{tikzcd}\]
Since this diagram commutes, it is sensible to say that a wedge associates to every $f \in \Hom_\C(C, C')$ a map $H(\Id, f) \circ \eta_C = H(f, \Id) \circ \eta_{C'} : X \to H(C, C')$.
In other words, we have a collection of functions $\eta_{C, C'} : \Hom_\C(C, C') \to \Hom_{\Set}(X, H(C, C'))$.
Functoriality of $H$ assures us that the diagram below commutes.
% https://q.uiver.app/#q=WzAsNCxbMCwwLCJcXEhvbV9cXEMoQywgQycpIl0sWzEsMCwiXFxIb21fXFxDKEMsIEMnJykiXSxbMCwxLCJcXEhvbV97XFxTZXR9KFgsIEgoQywgQycpKSJdLFsxLDEsIlxcSG9tX3tcXFNldH0oWCwgSChDLCBDJycpKSJdLFswLDEsIlxcSG9tX1xcQyhDLCBnKSJdLFsyLDMsIlxcSG9tX3tcXFNldH0oWCwgSChDLCBmKSkiLDIseyJvZmZzZXQiOjIsInN0eWxlIjp7ImJvZHkiOnsibmFtZSI6Im5vbmUifSwiaGVhZCI6eyJuYW1lIjoibm9uZSJ9fX1dLFswLDIsIlxcZXRhX3tDLCBDJ30iLDJdLFsxLDMsIlxcZXRhX3tDLCBDJyd9Il0sWzIsM11d
\[\begin{tikzcd}
	{\Hom_\C(C, C')} & {\Hom_\C(C, C'')} \\
	{\Hom_{\Set}(X, H(C, C'))} & {\Hom_{\Set}(X, H(C, C''))}
	\arrow["{\Hom_\C(C, g)}", from=1-1, to=1-2]
	\arrow["{\eta_{C, C'}}"', from=1-1, to=2-1]
	\arrow["{\eta_{C, C''}}", from=1-2, to=2-2]
	\arrow["{\Hom_{\Set}(X, H(C, f))}"', shift right=2, draw=none, from=2-1, to=2-2]
	\arrow[from=2-1, to=2-2]
\end{tikzcd}\]
There is a similar diagram for the first argument, and together these amount to naturality of the family $\eta_{C, C'}$ with respect to $\C^\op \times \C$.
That is, a wedge from $X$ into $H$ is equivalently a natural transformation from $\Hom_\C(-, -)$ to $\Hom_{\Set}(X, H(-, -))$.

\begin{definition}
  The \textit{end} of $H : \C^\op \times \C \to \Set$ is, if it exists, the set $\int_{\C} H$ such that there are isomorphisms, natural in $X$, of the following form:
  \[
    \left[ X, \int_{\C} H \right]
    \cong [\C^\op \times \C, \Set](\Hom_\C(-, -), \Hom_{\Set}(X, H(-, -))).
  \]
\end{definition}
Though ends aren't always equalizers, they usually have some sort of limit formula.
For instance, in $\infty$-categories, ends can be computed as limits over cosimplicial objects.

\subsection{An aside on the Yoneda embedding}
The tool that allows us to assert the well-definedness of the definition above is the Yoneda embedding.

\begin{definition}
  Let $\C$ be a category.
  The Yoneda embedding is the functor
  \begin{gather*}
    y : \C \to [\C^\op, \Set],\\
    C \mapsto \Hom_\C(-, C).
  \end{gather*}
  Equivalently, this functor is the image of the $\Hom_\C$ functor under the adjunction equivalence
  \[ [\C^\op \times \C, \Set] \simeq [\C, [\C^\op, \Set]]. \]
\end{definition}

This functor is fully faithful (the inverse is given by sending a natural transformation $\eta : \Hom_\C(-, X) \to \Hom_\C(-, Y)$ to $\eta_X(\Id_X)$).
This is unbelievably useful.
One such application is a proof that the centre of a category is an end:
\begin{align*}
  [\C, \C](\Id_\C, \Id_\C)
  &\xrightarrow{y_{\C\ast}} [\C, [\C^\op, \Set]](y_\C, y_\C)\\
  &\xrightarrow{(-)^\flat} [\C^\op \times \C, \Set](\Hom_\C, \Hom_\C)\\
  &\simeq [\C^\op \times \C, \Set](\Hom_\C(-, -), \Hom_{\Set}(*, \Hom_\C(-, -)))\\
  &\simeq \left[\ast, \int_\C \Hom_\C \right]\\
  &\simeq \int_\C \Hom_\C.
\end{align*}

\subsection{Monadicity and the bimodule centre}
The previous section also includes a proof that the bimodule centre agrees with the centre of a category, but this is obscured because it involves thinking of the bimodule category from a different perspective.

The reader will most likely be familliar with the following phenomenon.
For algebraic objects (e.g.\ groups, rings, monoids) isomorphisms can be checked on the underlying set, and the group/ring/monoid structure is automatically prserved.
On the other hand, there are bijections of geometric objects (e.g.\ topological spaces) which are not isomorphisms.
This is captured by the idea of \textit{monadicity}.

\begin{definition}
  Let $L : \C \rightleftarrows \D : R$ be an adjunction.
  The adjunction is \textit{monadic} if
  \begin{itemize}
    \item $R$ reflects isomorphisms, and
    \item if, for every $f, g : C \to C'$ such that $U(f), U(g)$ admits a split coequalizer in $\C$, the maps $f, g$ have a coequalizer in $\D$, and $R$ preserves this coequalizer.
  \end{itemize}
\end{definition}
A \textit{split coequalizer} is a coequalizer as in the following diagram below,
% https://q.uiver.app/#q=WzAsMyxbMCwwLCJDIl0sWzIsMCwiRCJdLFs0LDAsIlgiXSxbMCwxLCJmIiwwLHsib2Zmc2V0IjotMn1dLFswLDEsImciLDIseyJvZmZzZXQiOjJ9XSxbMSwyLCJjIiwyXSxbMiwxLCJiIiwyLHsiY3VydmUiOjMsInN0eWxlIjp7ImJvZHkiOnsibmFtZSI6ImRhc2hlZCJ9fX1dLFsxLDAsImEiLDIseyJjdXJ2ZSI6Mywic3R5bGUiOnsiYm9keSI6eyJuYW1lIjoiZGFzaGVkIn19fV1d
\[\begin{tikzcd}
	C && D && X
	\arrow["f", shift left=2, from=1-1, to=1-3]
	\arrow["g"', shift right=2, from=1-1, to=1-3]
	\arrow["a"', curve={height=18pt}, dashed, from=1-3, to=1-1]
	\arrow["c"', from=1-3, to=1-5]
	\arrow["b"', curve={height=18pt}, dashed, from=1-5, to=1-3]
\end{tikzcd}\]
where $b;c = \Id_X$, $c;b = a;g$ and $f;a = \Id_D$.

If an adjunction is monadic, then the source of the right adjoint is the category of modules over the monad $R \circ L$ associated to the adjunction.
This gives us a way of proving that certain categories are equivalent without directly constructing an equivalence.

In particular, the free/forgetful adjunction
\[
  U : \text{Bimod}_{A|A} \rightleftarrows \Set : F
\]
and the left Kan extension/precomposition adjunction induced by $\ast \to BA^\op \times BA$
\[
  (\bullet, \bullet)^* : [BA^\op \times BA, \Set] \rightleftarrows [*, \Set] : (\bullet, \bullet)_!
\]
are both monadic, and the monads are equivalent.
Therefore the categories are equivalent.
Under this equivalence, the $A|A$-bimodule $A$ is taken to $\Hom_{BA}$, so the equivalence from the previous section factors through the bimodule centre (the third line).

\section{Morita invariance}
You can also get morita invariance of the centre from the perspective of ends.
Because $[BA, \Set]$ is the category of left $A$-modules (again, one can use monadicity), we have
\begin{align*}
  &[[BA, \Set], [BA, \Set]](\Id_{[BA, \Set]}, \Id_{[BA, \Set]})\\
  &\xrightarrow{y_{BA^\op}^\ast} [BA^\op, [BA, \Set]](y_{BA^\op}, y_{BA^\op})\\
  &\xrightarrow{(-)^{\flat}} [BA^\op \times BA, \Set](\Hom_{BA}, \Hom_{BA}),
\end{align*}
this final set being the bimodule centre.

\bibliographystyle{alpha}
\bibliography{refs}

\end{document}